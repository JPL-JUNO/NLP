\chapter{Packets of thought(NLP overview)\label{Ch01}}
\section{自然语言与编程语言}

\begin{definition}
    自然语言处理是计算机科学和人工智能(artificial intelligence,AI)的
    一个研究领域,它关注自然语言(如英语或汉语普通话)的处理。这种处理
    通常包括将自然语言转换成计算机能够用于理解这个世界的数据(数字)。
    同时,这种对世界的理解有时被用于生成能够体现这种理解的自然语言文本
    (即自然语言生成)。
\end{definition}

\section{计算机“眼”中的语言}
\subsection{一个简单的聊天机器人}

受计算资源所限,早期的NLP研究人员不得不使用人类大脑的计算能力
来设计和手动调整复杂的逻辑规则来从自然语言字符串中提取信息。这称为
\textbf{基于模式(pattern)}的NLP方法。这些模式就像正则表达式那样,可以不仅
仅是字符序列模式。NLP还经常涉及词序列、词性或其他高级的模式。核心
的NLP构建模块(如词干还原工具和分词器)以及复杂的端到端NLP对话引
擎(聊天机器人)(如ELIZA)都是通过这种方式,即基于正则表达式和模
式匹配来构建的。基于模式匹配NLP方法的艺术技巧在于,使用优雅的模式
来获得想要的内容,而不需要太多的正则表达式代码行。

\begin{tcolorbox}[title=经典心智计算理论]
    这种经典的NLP模式匹配方法是建立在心智计算理论(computional
    theory of mind,CTM)的基础上的。CTM假设类人NLP可以通过一系列处理
    的有限逻辑规则集来完成。在世纪之交,神经科学和NLP的进步导致了心
    智“连接主义”理论的发展,该理论允许并行流水线同时处理自然语言,就像
    在人工神经网络中所做的那样。
\end{tcolorbox}
\subsection{另一种方法}
为数值序列和向量设计的距离度量方法对一些NLP应用程序来说非常有
用,如拼写校正器和专有名词识别程序。所以,当这些距离度量方法有意义
时,我们使用这些方法。但是,针对那些我们对自然语言的含义比对拼写更
感兴趣的NLP应用程序来说,有更好的方法。对于这些NLP应用程序,我们
使用自然语言词和文本的向量表示以及这些向量的一些距离度量方法。

\figures{fig1-2}{词条分拣托盘}

\section{聊天机器人的自然语言流水线}
聊天机器人需要4个处理阶段和一个数据库来维护过去语句和回复的记
录。这4个处理阶段中的每个阶段都可以包含一个或多个并行或串行工作的
处理算法(如 \autoref{fig1-3} 所示)。
\begin{enumerate}
    \item 解析:从自然语言文本中提取特征、结构化数值数据。
    \item 分析:通过对文本的情感、语法合法度及语义打分,生成和组合特征。
    \item 生成:使用模板、搜索或语言模型生成可能的回复。
    \item 执行:根据对话历史和目标,规划相应语句,并选择下一条回复。
\end{enumerate}

\figures{fig1-3}{Chatbot recirculating (recurrent) pipeline}