\chapter{构建自己的词汇表-分词\label{Ch02}}
\section{挑战(词干还原预览)}
所谓词干还原,指的是将某个词的不同屈折变化形式统统“打包”到同一个“桶”或者类别中。
\section{利用分词器构建词汇表}

\subsection{点积}
点积也称为内积(inner product),这是因为两个向量(每个向量中的元素个数)或矩阵(第一个矩阵的行数和第二个矩阵的列数)的“内部”维度必须一样,这种情况下才能相乘。这和关系数据库表的内连接(innerjoin)操作很类似。

点积也称为标积(scalar product),因为其输出结果是个单独的标量值。这使其有别于叉积(cross product)这个概念,后者的输出结果是一个向量。

\begin{tcolorbox}
    点积和矩阵乘积(matrix product)计算是等价的,后者可以在numpy中
    使用np.matmul()函数或者@操作符来实现。由于所有的向量都可以转换成
    $N \times 1$或者$1 \times N$的矩阵,因此可以使用这种短记号作用于两个列向量($N\times1$),其中第一个向量必须要转置,这样才可以相乘。
\end{tcolorbox}
\subsection{度量词袋之间的重合度}
如果能够度量两个向量词袋之间的重合度,就可以很好地估计它们所用词的相似程度,而这也是它们语义上重合度的一个很好的估计。
\subsection{}
\subsection{}